%----------------------------------------------------------------------------------------
%	SECTION TITLE
%----------------------------------------------------------------------------------------

\cvsection{Professional Experience}

%----------------------------------------------------------------------------------------
%	SECTION CONTENT
%----------------------------------------------------------------------------------------

\begin{cventries}

%------------------------------------------------

\cventry
{Role : Web developer \& Teaching Assistant}
{Online course : Geometry of chaos}
{www.chaosbook.org/course1}
{2015 Spring}
{
  \begin{cvitems}
    \item {\textbf{Achievement : } Design and implement online autograder \& Design Homework for 16 weeks.}
    \item {\textbf{Core features:} Auto grade studensts' online submissions
        \quad\&\quad Email back grades automatically \quad\&\quad Provide a straightforward interface for the customer (the course instructor)
      to view the grades online.}
    \item {\textbf{Framework :} Django in Python, deployed in Heroku \quad
      \textbf{Repository :} {\color{black} https://github.com/dingxiong/phys7224} 
    }
  \end{cvitems}
}

\cventrymore
{\textbf{Research topic :} \emph{Computation of Floquet vectors in Kuramoto-Sivashinsky system}}
{Center for Nonlinear Science, Georgia Institute of Technology}
{Atlanta, GA, USA}
{2013 -- 2014}
{Role : Research Assistant \quad Adviser : Prof. Predrag Cvitanovi\'c }
{
  \begin{cvitems}
    % \item {\textbf{Adviser :} Prof. Predrag Cvitanovi\'c}
  \item {\textbf{Main result:} The Floquet multipliers of Periodic orbits in high dimensional system 
      usually spans a large orders of magnitudes. The periodic eigendecomposition is the right tool
      to obtain Floquet spectrum and vectors to high accuracy. See paper[2] for more detail.
    }
  \end{cvitems}
}

\cventry
{\textbf{Research topic :} \emph{Investigation of the local dimension of inertial manifolds in chaotic systems}}
{}{}
{2014 -- 2015}
{
  \begin{cvitems}
    %\item {\textbf{Adviser :} Prof. Predrag Cvitanovi\'c}
    \item {\textbf{Main result:} By studying the shadowing cases of periodic orbits in
        Kuramoto-Sivashinsky system, we show strong evidence that the inertial manifold has
        dimension 8. see paper [1] for more details.}
  \end{cvitems}
}

\cventry
{\textbf{Research topic :} \emph{Symbolic dynamics in symmetry reduced 1-d Kuramoto-Sivashinsky system}}
{}{}
{2015 -- Present}
{
  \begin{cvitems}
    %\item {\textbf{Adviser :} Prof. Predrag Cvitanovi\'c}
    %\item {\textbf{Goal:} In progress}
      \item {In the symmetry reduced state space, the attractor of  1-d Kuramoto-Sivashinsky system
          is low dimensional. By constructing appropriate Poincar\'e section, we propose to obtain the
          symbolic dynamics. }
  \end{cvitems}
}


\cventrymore
{\textbf{Research topic :} \emph{Integration of soliton explosion with local error control in cubic quintic 
  Ginzburg-Landau system}}
{School of Mathematics, Georgia Tech} 
{Atlanta, GA, USA} % Location
{Sprint 2016} % Date(s)
{Role : Cooperation with Prof. Sung Ha Kang from Math department}
{
  \begin{cvitems}
  \item {\textbf{Main result:}  Study the performance of exponential integrator in Ginzburg-Landau system,
      and add time step control into a few popular exponential integrators. See paper [3].}
  \end{cvitems}
}


% \cventry 
% {Teaching Assistant}
% {Center for Nonlinear Science, School of Physics,
%   Georgia Institute of Technology}
% {Atlanta, GA, USA}
% {2012 -- 2013}
% {
%   \begin{cvitems}
%   \item { Guide undergraduate students in physical experiments.}
%   \end{cvitems}
% }


%------------------------------------------------

\end{cventries}
