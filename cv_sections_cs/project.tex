%----------------------------------------------------------------------------------------
%	SECTION TITLE
%----------------------------------------------------------------------------------------

\cvsection{Selected CS Course Projects}

%----------------------------------------------------------------------------------------
%	SECTION CONTENT
%----------------------------------------------------------------------------------------

\begin{cventries}

% ------------------------------------------------

\cventry
{\textbf{Project :} Triangular Mesh} 
{Course project: Gatech CS6491 Computer Graphics} 
{Atlanta, GA, USA} % Location
{Fall 2015} % Date(s)
{
  \begin{cvitems}
  \item {\textbf{Achievement :} Represent a triangular mesh by \textbf{CSX} table \quad\&\quad Implement navigating
      methods on the mesh, such as swing, opposite, left, right and so on
      \quad\&\quad  Solve 4 interesting problems: \textbf{geodesic path}, 
      \textbf{Gaussian decay swirl}, \textbf{lasso deletion} and \textbf{mesh cut}.}
  \item {\textbf{Toolbox :} Processing \quad
      \textbf{Repository :} {\color{red} https://github.com/dingxiong/triangularMesh}
      \quad
      \textbf{Demo : }  {\color{red} https://youtu.be/mWe0YO1bbZ4}}
  \end{cvitems}
}

\cventry
{\textbf{Project :} RPC-Based Proxy Server} 
{Course project: Gatech CS6210 Advanced Operating System} 
{Atlanta, GA, USA} % Location
{Spring 2015} % Date(s)
{
  \begin{cvitems}
  \item {\textbf{Achievement:} Build a proxy server by remote procedure call(\textbf{RPC}) 
\quad\&\quad test the performance of four different cache polices : no cache, Least Recent Used (LUR), random,
      and First in First out (FIFO). 
\quad\&\quad RPC framework is provided by \textbf{Apache Thrift} library.
    }
  \item {\textbf{Language :} C++  \quad
      \textbf{Repository :} {\color{red} https://github.com/dingxiong/CS6210Project3}
    }
  \end{cvitems}
}

\cventry
{\textbf{Project :} Cache Design for Four Different Traces} 
{Course project: Gatech CS6290 High Performance Computer Architecture} 
{Atlanta, GA, USA} % Location
{Summer 2014} % Date(s)
{
  \begin{cvitems}
  \item {\textbf{Achievement :} Design and implement a parametric cache
      simulator \quad\&\quad Design data caches well suited to the SPEC benchmarks. 
      \quad\&\quad Optimize cache with respect to variables
      including $2^C$ bytes of cache size, $2^S$ blocks with each block $2^{C-S}$ bytes,
      storage policies (ST) and replacement policies (R).
    }
  \item {\textbf{Language :} C++  \quad
      \textbf{Repository :} {\color{red} https://github.com/dingxiong/cacheDesign}
    }
  \end{cvitems}
}

\cventry
{\textbf{Project :} CPU and GPU optimization in finding initial 
  condition for Kuramoto Sivashinsky equation} 
{Course project: Gatech CSE6230 High Performance Computing : Tools and Applications} 
{Atlanta, GA, USA} % Location
{Fall 2013} % Date(s)
{
  \begin{cvitems}
  \item {\textbf{main goal :} CPU and GPU optimization is deployed to find relative 
      good initial conditions for Kuramoto-Sivashinsky equation. 
    }
    \item {\textbf{Achievement: }Our result shows that the 
      \textbf{icc \& Cilk} approach has the best performance of all multi CPU implementation, and
      the GPU implementation has better performance if register usage is considered.
    }
  \item {\textbf{Language :} C \quad 
      \textbf{Tools : } gcc, icc, OpenMP, Cilk, CUDA, SIMD(SSE2, SSE4)
    }
  \item {
      \textbf{Repository :} {\color{red} https://bitbucket.org/dingxiong/project}
    }
  \end{cvitems}
}


% \cventry
% {\textbf{Systems : } Kuramoto-Sivashinsky equation and complex cubic quintic Ginzburg-Landau equation} 
% {Research code: Nonlinear dynamics} 
% {Atlanta, GA, USA} % Location
% {2013 -- PRESENT} % Date(s)
% {
%   \begin{cvitems}
%   \item {\textbf{Languages :} C++, Python, Matlab \quad
%       \textbf{Tools :} Boost.Python, Boost.Numpy, HDF5, Arpack
%     }
%   \item {\textbf{Repository :} {\color{red} https://github.com/dingxiong/research}}
%   \end{cvitems}
% }


%------------------------------------------------


%------------------------------------------------

\end{cventries}