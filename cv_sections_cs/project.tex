%----------------------------------------------------------------------------------------
%	SECTION TITLE
%----------------------------------------------------------------------------------------

\cvsection{Projects}

%----------------------------------------------------------------------------------------
%	SECTION CONTENT
%----------------------------------------------------------------------------------------

\begin{cventries}

\cvsubsection{Computer Science related}
% ------------------------------------------------
\cventry
{\textbf{Project :} Online autograder} 
{Online course : Geometry of Chaos} 
{Atlanta, GA, USA} % Location
{Sprint 2015} % Date(s)
{
  \begin{cvitems}
  \item {\textbf{main goal :} Auto grade studensts' online submissions and email back their grades. Also, 
      it provides a straightforward interface for the customer (the course instructor)
      to view the grades online.}
  \item {\textbf{Framework :} Django in Python, deployed in Heroku \quad
      \textbf{Repository :} {\color{red} https://github.com/dingxiong/phys7224} 
    }
  \end{cvitems}
}

\cventry
{\textbf{Project :} Triangular Mesh} 
{Course project: Gatech CS6491 Computer Graphics} 
{Atlanta, GA, USA} % Location
{Fall 2015} % Date(s)
{
  \begin{cvitems}
  \item {\textbf{main goal :} Represent a triangular mesh by \textbf{CSX} table, and implement navigating
      methods on the mesh, such as swing, opposite, left, right and so on. Also 4 interesting problems
      are defined and solved: \textbf{geodesic path}, 
      \textbf{Gaussian decay swirl}, \textbf{lasso deletion} and \textbf{mesh cut}.}
  \item {\textbf{Toolbox :} Processing \quad
      \textbf{Repository :} {\color{red} https://github.com/dingxiong/triangularMesh}
      \quad
      \textbf{Demo : }  {\color{red} https://youtu.be/mWe0YO1bbZ4}}
  \end{cvitems}
}

\cventry
{\textbf{Project :} RPC-Based Proxy Server} 
{Course project: Gatech CS6210 Advanced Operating System} 
{Atlanta, GA, USA} % Location
{Spring 2015} % Date(s)
{
  \begin{cvitems}
  \item {\textbf{main goal :} Build a proxy server by remote procedure call(\textbf{RPC}) and test
      the performance of four different cache polices : no cache, Least Recent Used (LUR), random,
      and First in First out (FIFO). RPC framework is provided by \textbf{Apache Thrift} library.
    }
  \item {\textbf{Language :} C++  \quad
      \textbf{Repository :} {\color{red} https://github.com/dingxiong/CS6210Project3}
    }
  \end{cvitems}
}

\cventry
{\textbf{Project :} Cache Design for Four Different Traces} 
{Course project: Gatech CS6290 High Performance Computer Architecture} 
{Atlanta, GA, USA} % Location
{Summer 2014} % Date(s)
{
  \begin{cvitems}
  \item {\textbf{main goal :} Design and implement a parametric cache
      simulator and use it to design data caches well suited to the SPEC benchmarks. 
      Variables include $2^C$ bytes of cache size, $2^S$ blocks with each block $2^{C-S}$ bytes,
      storage policies (ST) and replacement policies (R).
    }
  \item {\textbf{Language :} C++  \quad
      \textbf{Repository :} {\color{red} https://github.com/dingxiong/cacheDesign}
    }
  \end{cvitems}
}

\cventry
{\textbf{Project :} CPU and GPU optimization in finding initial 
  condition for Kuramoto Sivashinsky equation} 
{Course project: Gatech CSE6230 High Performance Computing : Tools and Applications} 
{Atlanta, GA, USA} % Location
{Fall 2013} % Date(s)
{
  \begin{cvitems}
  \item {\textbf{main goal :} CPU and GPU optimization is deployed to find relative 
      good initial conditions for Kuramoto-Sivashinsky equation. 
    }
    \item {\textbf{result : }Our result shows that the 
      \textbf{icc \& Cilk} approach has the best performance of all multi CPU implementation, and
      the GPU implementation has better performance if register usage is considered.
    }
  \item {\textbf{Language :} C \quad 
      \textbf{Tools : } gcc, icc, OpenMP, Cilk, CUDA, SIMD(SSE2, SSE4)
    }
  \item {
      \textbf{Repository :} {\color{red} https://bitbucket.org/dingxiong/project}
    }
  \end{cvitems}
}


% \cventry
% {\textbf{Systems : } Kuramoto-Sivashinsky equation and complex cubic quintic Ginzburg-Landau equation} 
% {Research code: Nonlinear dynamics} 
% {Atlanta, GA, USA} % Location
% {2013 -- PRESENT} % Date(s)
% {
%   \begin{cvitems}
%   \item {\textbf{Languages :} C++, Python, Matlab \quad
%       \textbf{Tools :} Boost.Python, Boost.Numpy, HDF5, Arpack
%     }
%   \item {\textbf{Repository :} {\color{red} https://github.com/dingxiong/research}}
%   \end{cvitems}
% }


\cvsubsection{Physics related}
%------------------------------------------------

\cventry
{\textbf{Project :} Computation of Floquet vectors in Kuramoto-Sivashinsky system} 
{Center for Nonlinear Science, School of Physics, Georgia Tech} 
{Atlanta, GA, USA} % Location
{2013-2014} % Date(s)
{
  \begin{cvitems}
    %\item {\textbf{Adviser :} Prof. Predrag Cvitanovi\'c}
    \item {\textbf{Main result:} The Floquet multipliers of Periodic orbits in high dimensional system 
        usually spans a large orders of magnitudes. The periodic eigendecomposition is the right tool
        to obtain Floquet spectrum and vectors to high accuracy. See paper[2] for more detail.
      }
  \end{cvitems}
}

\cventry
{\textbf{Project :} Investigation of the local dimension of inertial manifolds in chaotic systems} 
{Center for Nonlinear Science, School of Physics, Georgia Tech} 
{Atlanta, GA, USA} % Location
{2014-2015} % Date(s)
{
  \begin{cvitems}
    %\item {\textbf{Adviser :} Prof. Predrag Cvitanovi\'c}
    \item {\textbf{Main result:} By studying the shadowing cases of periodic orbits in
        Kuramoto-Sivashinsky system, we show strong evidence that the inertial manifold has
        dimension 8. see paper [1] for more details.}
  \end{cvitems}
}

\cventry
{\textbf{Project :} Symbolic dynamics in symmetry reduced 1-d Kuramoto-Sivashinsky system} 
{Center for Nonlinear Science, School of Physics, Georgia Tech} 
{Atlanta, GA, USA} % Location
{2015-PRESENT} % Date(s)
{
  \begin{cvitems}
    %\item {\textbf{Adviser :} Prof. Predrag Cvitanovi\'c}
    %\item {\textbf{Goal:} In progress}
      \item {In the symmetry reduced state space, the attractor of  1-d Kuramoto-Sivashinsky system
          is low dimensional. By constructing appropriate Poincar\'e section, we propose to obtain the
          symbolic dynamics. }
  \end{cvitems}
}

\cvsubsection{Mathematics related}

\cventry
{\textbf{Project :} Integration of soliton explosion with local error control in cubic quintic 
  Ginzburg-Landau system} 
{School of Mathematics, Georgia Tech} 
{Atlanta, GA, USA} % Location
{Sprint 2016} % Date(s)
{
  \begin{cvitems}
  \item {\textbf{Adviser :} Prof. Sung Ha Kang}
  \item {\textbf{Main result:}  add time step control into a few popular exponential integrators. See paper [3].}
  \end{cvitems}
}




%------------------------------------------------


%------------------------------------------------

\end{cventries}