%---------------------------------------------------------------------------
%	SECTION TITLE
%---------------------------------------------------------------------------

\cvsection{Research Experience}

%--------------------------------------------------------------------------
%	SECTION CONTENT
%---------------------------------------------------------------------------

\begin{cventries}
  % ------------------------------------------------
  \cvresearch
  {Center for Nonlinear Science, School of Physics, Georgia Institute of Technology}
  {Atlanta, GA, USA}
  {Jun. 2013 -- May. 2017}
  {
    \cvresearchitem
    {Computation of Floquet vectors in Kuramoto-Sivashinsky system}
    {The Floquet multipliers of Periodic orbits in high dimensional systems usually spans a large orders 
     of magnitudes. The periodic eigendecomposition is the right tool to obtain Floquet spectrum and
     vectors to high accuracy, by which We find the smallest eigenvalue of Floquet matrix to be order of 
     $10^{-3000}$ with relative accuracy $10^{-14}$. See paper[2] for more detail.}
    {C++, Boost.Python, Boost.Numpy, HDF5, Arpack, Matrix decomposition, Eigen}
    \cvresearchitem
    {Investigation of the local dimension of inertial manifolds in chaotic systems}
    {By studying the shadowing cases of periodic orbits in one-dimensional Kuramoto-Sivashinsky system, 
     we show strong evidence that its inertial manifold at domain size 22 has dimension 8. see paper [1] for more details. }
    {C++, Matlab, Exponential integrators}
    \cvresearchitem
    {Symbolic dynamics in symmetry reduced 1-d Kuramoto-Sivashinsky system}
    {In the symmetry reduced state space, we propose to obtain the symbolic dynamics of 1-d KS equation by constructing appropriate Poincar\'e sections.}
    {C++, Matlab, Eigen, Cycle expansion theory}
  }

  \cvresearch
  {School of Mathematics, Georgia Institute of Technology} 
  {Atlanta, GA, USA}
  {Jan. 2016 -- Jun. 2016} 
  {
    \cvresearchitem
    {Time-step adaptive exponential integrator for soliton explosions in 1d and 2d cubic quintic Ginzburg-Landau systems}
    {Formulize a new time-step adaptive exponential integrator for complex GL equation, which
     substantially slows down the integration of the soliton explosion part. See paper[3] for more
     detail.}
    {Numerical PDE, C++, Boost, Numpy, Matplotlib}
  }

%------------------------------------------------

\end{cventries}
