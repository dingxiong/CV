%---------------------------------------------------------------------------
%	SECTION TITLE
%---------------------------------------------------------------------------

\cvsection{Research Experience}

%--------------------------------------------------------------------------
%	SECTION CONTENT
%---------------------------------------------------------------------------

\begin{cventries}
  % ------------------------------------------------
  \cvresearch
  {Center for Nonlinear Science, School of Physics, Georgia Institute of Technology}
  {Atlanta, GA, USA}
  {Jun. 2013 -- May. 2017}
  {Role : Research Assistant \quad Adviser : Prof. Predrag Cvitanovi\'c }
  {
    \cvresearchitem
    {Computation of Floquet vectors in Kuramoto-Sivashinsky system}
    {The Floquet multipliers of Periodic orbits in high dimensional system usually spans a large orders of magnitudes. The periodic eigendecomposition is the right tool to obtain Floquet spectrum and vectors to high accuracy. See paper[2] for more detail.}
    \cvresearchitem
    {Investigation of the local dimension of inertial manifolds in chaotic systems}
    {By studying the shadowing cases of periodic orbits in        Kuramoto-Sivashinsky system, we show strong evidence that the inertial manifold has dimension 8. see paper [1] for more details. }
    \cvresearchitem
    {Symbolic dynamics in symmetry reduced 1-d Kuramoto-Sivashinsky system}
    {In the symmetry reduced state space, the attractor of  1-d Kuramoto-Sivashinsky system is low dimensional. By constructing appropriate Poincar\'e section, we propose to obtain the symbolic dynamics.}
  }

  \cvresearch
  {School of Mathematics, Georgia Institute of Technology} 
  {Atlanta, GA, USA}
  {Jan. 2016 -- Jun. 2016} 
  {Role : Cooperation with Prof. Sung Ha Kang from Math department}
  {
    \cvresearchitem
    {Time-step adaptive exponential integrator for soliton explosions in 1d and 2d cubic quintic Ginzburg-Landau systems}
    {Study the performance of exponential integrator in Ginzburg-Landau system,      and add time step control into a few popular exponential integrators. See paper [3].}
  }

%------------------------------------------------

\end{cventries}
